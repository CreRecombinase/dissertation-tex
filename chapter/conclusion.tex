\chapter{Conclusion}\label{conclusion}

The classical definition of a gene is a "unit of heredity" \cite{gene_book}. Historically,
the gene was defined as that which gave rise to (i.e caused) that which was heritable.  
According to the classical definition, the notion of a "functional gene" is purely redundant.
Inherent then to the idea of the gene is that it is functionional.
The job of the biologist is to characterize the diversity of living organisms, the physical, chemical and the molecular basis of that activity, and the ways those parts relate to one-another.


Fine mapping can be considered on several levels.
On a statistical level, we reintroduce the multivariate regression concept.
In a Bayesian approach, we model the observed univariate effect sizes as coming from some unobserved distribution of true effects,
that are then obscured by noise and LD.
On a biological level, a locus has been fine mapped if consequences of genetic modification with respect to the trait of
interest can be accurately predicted the mechanism underlying the statistical association at the locus can be explained.

Rather than a single link between variant and trait, we can construct a chain of causality from variant to molecular intermediate, and then from molecular intermediate to trait
of interest.  In the case of molecular intermediates, there are two broad categories of variant to molecular-intermediate causal relationships that are especially tractable for 
validation.  By virtue of the mapping of the human genome and our knowledge of the central dogma of molecular biology cite\textsubscript{crick1958protein} , by knowing the position and sequence
of a variant one can immediately know with the variant lies within an exon (or splice-site boudary) of a characterized gene.  If so, it is a simple bioinformatic exercise to 
ascertain the consequence of the change in DNA sequence on primary (and increasingly secondary and tertiary) amino acid structure.  Characterizing how changes in protein
structure contribute to changes in protein function is a sub-discipline of biology all its own; but this is a question of "how" a variant-gene pair interact, and not a question of 
"if" they interact.  The functional consequence of the majority of human genetic variation falls into the second category of variant to molecular-intermediate causal relationships.  
This second category consists of all other mechanisms of causality, perhaps the best understood of which are the "regulatory variants".

In this dissertation I have examined the relationship between phenotype and genotype from three perspectives.  With RSSp I explored the phenotype genotype relationship at its most diffuse.
I demonstrated how the infinitesimal model --- a statistical model of inheritance that dates back over a century --- remains highly relevant in relating phenotype to genotype.

Finally, I would be remiss to not discuss how fraught questions of heritability may be.  Racists have used quantitative genetics as a cudgel since it's inception.  
As much as the field of quantitative genetics has advanced in the century since RA Fisher first described the infinitesimal model, there is in
fact very little about RSSp that would be foreign to him --- if we as geneticists have left behind the ghosts of our past, it is not on behalf of our methodlogical advances.
We as humans are fortunate that the worthiness of individuals of our species on account of their ancestry not an empirical question.  On this earth we simply have humans and everyone else.
RSSp works on GWAS summary statistics.  This means that the quality of the heritability inference is almost entirely reliant on the quality of the summary statistics.  If the original study suffered from population stratification, that bias will propagate into the heritability estimates of RSSp.  Of greater concern for me is the extent to which Europeans have become a de facto ``model organism'' in statistical genetics, and how little my work has done to push in the opposite direction.  I have built a tool that 1) is most useful in combination with a large ``reference'' LD panel that is as similar to the original dataset as possible and 2) provides no functionality for testing whether the reference LD panel is appropriate for the GWAS.
Without
The European individuals from the 1000 genomes project are a common source of LD information.  This is despite the fact that the GWAS datasets they are often used as reference LD panels for are very often orders of magnitude larger.  




