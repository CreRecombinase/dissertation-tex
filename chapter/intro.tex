\chapter{Introduction}
\label{introduction}

This is the introduction for my dissertation


In evaluating the hypothesis that a gene is a driver gene, in addition to considering the experimental evidence (i.e the likelihood of the somatic mutation data under this hypotheses as compared to the null hypothesis that it is not a driver gene), it can be very useful to consider the \emph{a priori} probability that the gene is a driver gene, condtional on the information that is known about that gene.  By combining \emph{a priori} information with the information from the statistical model,in the form of the likelihood (ratio), genes that have high statistical evidence of being driver genes but low prior probability of being driver genes can be down-weighted, while genes that have low or intermediate statistical evidence but high prior probability can be up-weighted.  Similarly, one can have higher certainty that genes with  both high statistical evidence and prior are driver genes and that genes with low statistical evidence and low prior probability are not driver genes.

Under the Bayesian paradigm, \emph{a posteriori} knowledge is reflected as \emph{a prioi} probability for future experiment.  A set of Bayes factors from a study of cancer driver genes identified using the \texttt{driverMAPS} method\cite{drivermaps}.  From 734,754 single nucleotide somatic mutations from 20 tumor types in TCGA, data from approximately 20,000 genes were analyzed with the goal of identifying driver genes. For each gene analyzed in each cancer type, \texttt{driverMAPs} outputs the evidence that it is a driver gene as a Bayes factor.

\subsection{The unreasonable effectiveness of linear models}\label{sec:orgd56a398}

There is a tension in statistical molecular genetics between the generic linear nature of the statistical models used to predict phenotype from genotype, and the fundamentally non-linear nature of the molecular traits themselves.  Indeed, there is a 
rich tradition of mathematical biology outside of genetics for which linear systems are the exception rather than the rule.  When discussing the application of linear methods to non-linear systems, there is a common quotation by physicist Stanislaw Ulam \cite{Campbell_2004} :

\begin{quote}
Using a term like nonlinear science is like referring to the bulk of zoology as the study of non-elephant animals. --Stanislaw Ulam
\end{quote}

For geneticists, and statistical geneticists in particular, there is a powerful tendency towards linear models.  The central limit theorem being a commonly cited justification.  