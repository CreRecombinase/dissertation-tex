


  \section{Abstract}
  While a genetic component of preterm birth (PTB) has long been recognized and recently mapped by genome-wide association studies (GWAS), the molecular determinants underlying PTB remain elusive. This stems in part from an incomplete availability of comprehensive functional genomic annotations in human cell types relevant to pregnancy and PTB. We generated transcriptome (RNA-seq), epigenome (ChIP-seq of H3K27ac, H3K4me1 and H3K4me3 histone modifications), open chromatin (ATAC-seq) and chromatin interaction (promoter capture Hi-C) annotations of cultured primary decidua-derived mesenchymal stromal/stem cells (MSCs) and in vitro differentiated decidual stromal cells (DSCs) and developed a computational framework to integrate these functional annotations with results from a GWAS of gestational duration in 56,384 women. We found a significant enrichment of heritability estimates in functional noncoding regions in stromal cells and uncovered additional loci associated with gestational duration and target genes of associated loci. Our strategy illustrates how systematic functional annotations in pregnancy-relevant cell types aid in the experimental follow-up of GWAS for PTB and, likely, other pregnancy-related conditions.

 \section{Introduction}
Spontaneous preterm birth (PTB), defined as spontaneous labor and birth before 37 weeks of gestation, is associated with significant infant mortality and morbidity, as well as long-term health consequences into adulthood (1). A genetic component to PTB has long been recognized but the significant role of environmental factors and the etiologic heterogeneity of birth before 37 weeks (2-4) has made it challenging to discover genetic associations and causal genes. For example, recent genome-wide association studies (GWASs) of gestational duration in 43,568 women (3,331 with a preterm delivery) (5) and in 84,689 infants (4,775 born preterm) (6) reported six and one genome-wide significant associations, respectively, with gestational duration considered as a continuous variable. Three loci were also associated with PTB (defined as a categorical variable of birth) in the maternal GWAS (5), but no loci were associated with PTB in the infant GWAS (6). These studies highlight the challenges of such complex and multifactorial phenotypes, and the need for additional approaches to facilitate discovery of genes contributing to gestational duration and PTB. 
Integrating GWAS results with genomic and epigenomic annotations is a promising approach for assigning function to variants discovered by GWAS, as well as for identifying additional associations that do not reach stringent genome-wide significance threshold (7, 8). While large consortia (e.g., ENCODE (9), GTEx (10), Roadmap Epigenomics (11)) have generated annotations of putative functional elements and genetic variants for many human cell types and tissues, there is a remarkable absence in these databases for the cell types and tissues that are relevant to pregnancy in general and to PTB in particular. Because the regulation of transcription has strong cell type-specific components, and because annotations in disease-relevant tissues or cells tend to be most enriched among GWAS signals for those specific diseases (10, 12), follow-up studies of GWASs of pregnancy-associated conditions have been disadvantaged compared to most other complex diseases due to the paucity of functional annotations in cells relevant to pregnancy. To fill this gap in knowledge, we characterized the transcriptional and chromatin landscapes of cultured mesenchymal stromal/stem cells (MSCs) collected from human placental membranes and decidualized MSCs, also known as decidual stromal cells or DSCs. These cells play critical roles in promoting successful pregnancy, interfacing with fetal cells throughout pregnancy, and the timing of birth (13, 14). We then built a computational framework that integrated these decidua-derived stromal cell annotations with the results of a large GWAS of gestational duration to facilitate discovery of PTB genes. 
This integrated analysis revealed a significant enrichment of heritability estimates for gestational duration in decidua-derived stromal cell genomic regions marked by open chromatin or histone marks. Leveraging those functional annotations in a Bayesian statistical framework, we discovered additional loci associated with gestational duration and improved fine mapping in regions associated with gestational duration. Finally, using promoter capture Hi-C (pcHi-C), we linked functionally-annotated gestational age-associated variants to their putative target genes. More generally, these functional annotations should prove a valuable resource for studying other pregnancy-related conditions, such as preeclampsia and recurrent miscarriage, as well as conditions associated with endometrial dysfunction, such as endometriosis and infertility.